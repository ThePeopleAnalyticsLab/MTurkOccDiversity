\documentclass[english,man]{apa6}

\usepackage{amssymb,amsmath}
\usepackage{ifxetex,ifluatex}
\usepackage{fixltx2e} % provides \textsubscript
\ifnum 0\ifxetex 1\fi\ifluatex 1\fi=0 % if pdftex
  \usepackage[T1]{fontenc}
  \usepackage[utf8]{inputenc}
\else % if luatex or xelatex
  \ifxetex
    \usepackage{mathspec}
    \usepackage{xltxtra,xunicode}
  \else
    \usepackage{fontspec}
  \fi
  \defaultfontfeatures{Mapping=tex-text,Scale=MatchLowercase}
  \newcommand{\euro}{€}
\fi
% use upquote if available, for straight quotes in verbatim environments
\IfFileExists{upquote.sty}{\usepackage{upquote}}{}
% use microtype if available
\IfFileExists{microtype.sty}{\usepackage{microtype}}{}

% Table formatting
\usepackage{longtable, booktabs}
\usepackage{lscape}
% \usepackage[counterclockwise]{rotating}   % Landscape page setup for large tables
\usepackage{multirow}		% Table styling
\usepackage{tabularx}		% Control Column width
\usepackage[flushleft]{threeparttable}	% Allows for three part tables with a specified notes section
\usepackage{threeparttablex}            % Lets threeparttable work with longtable

% Create new environments so endfloat can handle them
% \newenvironment{ltable}
%   {\begin{landscape}\begin{center}\begin{threeparttable}}
%   {\end{threeparttable}\end{center}\end{landscape}}

\newenvironment{lltable}
  {\begin{landscape}\begin{center}\begin{ThreePartTable}}
  {\end{ThreePartTable}\end{center}\end{landscape}}

  \usepackage{ifthen} % Only add declarations when endfloat package is loaded
  \ifthenelse{\equal{\string man}{\string man}}{%
   \DeclareDelayedFloatFlavor{ThreePartTable}{table} % Make endfloat play with longtable
   % \DeclareDelayedFloatFlavor{ltable}{table} % Make endfloat play with lscape
   \DeclareDelayedFloatFlavor{lltable}{table} % Make endfloat play with lscape & longtable
  }{}%



% The following enables adjusting longtable caption width to table width
% Solution found at http://golatex.de/longtable-mit-caption-so-breit-wie-die-tabelle-t15767.html
\makeatletter
\newcommand\LastLTentrywidth{1em}
\newlength\longtablewidth
\setlength{\longtablewidth}{1in}
\newcommand\getlongtablewidth{%
 \begingroup
  \ifcsname LT@\roman{LT@tables}\endcsname
  \global\longtablewidth=0pt
  \renewcommand\LT@entry[2]{\global\advance\longtablewidth by ##2\relax\gdef\LastLTentrywidth{##2}}%
  \@nameuse{LT@\roman{LT@tables}}%
  \fi
\endgroup}


\ifxetex
  \usepackage[setpagesize=false, % page size defined by xetex
              unicode=false, % unicode breaks when used with xetex
              xetex]{hyperref}
\else
  \usepackage[unicode=true]{hyperref}
\fi
\hypersetup{breaklinks=true,
            pdfauthor={},
            pdftitle={Comparing MTurk and the US Population's Occupational Diversity},
            colorlinks=true,
            citecolor=blue,
            urlcolor=blue,
            linkcolor=black,
            pdfborder={0 0 0}}
\urlstyle{same}  % don't use monospace font for urls

\setlength{\parindent}{0pt}
%\setlength{\parskip}{0pt plus 0pt minus 0pt}

\setlength{\emergencystretch}{3em}  % prevent overfull lines

\ifxetex
  \usepackage{polyglossia}
  \setmainlanguage{}
\else
  \usepackage[english]{babel}
\fi

% Manuscript styling
\captionsetup{font=singlespacing,justification=justified}
\usepackage{csquotes}
\usepackage{upgreek}



\usepackage{tikz} % Variable definition to generate author note

% fix for \tightlist problem in pandoc 1.14
\providecommand{\tightlist}{%
  \setlength{\itemsep}{0pt}\setlength{\parskip}{0pt}}

% Essential manuscript parts
  \title{Comparing MTurk and the US Population's Occupational Diversity}

  \shorttitle{MTurk's Occupational Diversity}


  \author{Christopher M. Castille\textsuperscript{1}, Bodour H. Mahmoud\textsuperscript{2}, Rachel L. Williamson\textsuperscript{3}, John E. Buckner\textsuperscript{4}, \& James De Leôn\textsuperscript{5}}

  \def\affdep{{"", "", "", "", ""}}%
  \def\affcity{{"", "", "", "", ""}}%

  \affiliation{
    \vspace{0.5cm}
          \textsuperscript{1} Nicholls State University\\
          \textsuperscript{2} Villanova University\\
          \textsuperscript{3} University of Georgia\\
          \textsuperscript{4} AlixPartners\\
          \textsuperscript{5} APT Metrics  }

  \authornote{
    \newcounter{author}
    All correspondence should be sent to the lead author at
    \href{mailto:chris_castille@mac.com}{\nolinkurl{chris\_castille@mac.com}}.

                      Correspondence concerning this article should be addressed to Christopher M. Castille. E-mail: \href{mailto:chris_castille@mac.com}{\nolinkurl{chris\_castille@mac.com}}
                                                        }


  \abstract{We compare the occupational diversity of two convenient MTurk samples
against the broader US economy from which they have been indirectly
sampled. We focus on occupational diversity given a chief concern that
findings from OB/HR investigations relying on MTurk might not generalize
to the broader US economy . Additionally, as researchers may hope to
target specific populations (e.g., sales professionals) understanding
the extent to which these professionals are represented in the MTurk
population would be helpful. Therefore, we compare the occupational
diversity of MTurk against that described by the Bureau of Labor
Statistics, testing the null hypothesis that these populations are
equivalent in terms of occupational diversity. Across both samples,
which were obtained from studies conducted in 2015, we found differences
suggesting that the MTurk population is generally overrepresented by
white collar professionals (though this is an imperfect trend) and
generally underrepresented by blue collar workers.}
  \keywords{MTurk, Occupational Diversity \\

    \indent Word count: 3000
  }


\usepackage[titles]{tocloft}
\cftpagenumbersoff{table}
\renewcommand{\cfttabpresnum}{\itshape\tablename\enspace}
\renewcommand{\cfttabaftersnum}{.\space}
\setlength{\cfttabindent}{0pt}
\setlength{\cftafterloftitleskip}{0pt}
\settowidth{\cfttabnumwidth}{Table 10.\qquad}



\usepackage{amsthm}
\newtheorem{theorem}{Theorem}
\newtheorem{lemma}{Lemma}
\theoremstyle{definition}
\newtheorem{definition}{Definition}
\newtheorem{corollary}{Corollary}
\newtheorem{proposition}{Proposition}
\theoremstyle{definition}
\newtheorem{example}{Example}
\theoremstyle{remark}
\newtheorem*{remark}{Remark}
\begin{document}

\maketitle

\setcounter{secnumdepth}{0}



\section{Introduction}\label{introduction}

Amazon's Mechanical Turk (MTurk), an online labor market created by
Amazon, has recently become popular among social scientists interested
in collecting inexpensive, high-quality data from a diverse population
(Aguinis and Lawal (2012) Behrend, Sharek, Meade, and Wiebe (2011)
Bergman and Jean (2016) Paolacci and Chandler (2014) Shank (2016)).
Leveraging the strengths of this tool, MTurk has been used by social
scientists to replicate well-known phenomena (e.g., reaction times,
priming, task switching; Crump, McDonnell, and Gureckis (2013); memory,
Simons and Chabris (2012)) and by organizational behavior (OB)
researchers to expand the OB literature (e.g., Yarkoni, Ashar, and Wager
(2015)).

Yet, in spite of the growing enthusiasm surrounding the use of MTurk, it
represents merely another convenient sampling methodology that comes
with a few constraints, such as repeated participation or motivation due
to compensation (Landers and Behrend (2015) Woo, Keith, and Thornton
(2015)) and other yet-to-be-identified constraints. Here, we examine one
potentially important constraint for using MTurk to conduct OB research:
the occupational diversity of MTurk. Given that a fundamental goal of
most OB research is that workers are sampled from an appropriate
organizational setting (e.g., Pathak (2008)), understanding the
occupational diversity of MTurk will improve our understanding of the
boundary conditions or limits to the generalizability of MTurk research
findings. By examining the occupational diversity of MTurk, our work
addresses a call for research that describes the kinds of individuals
who participate in MTurk studies (Griggs et al. (2016)).

\section{Review of the Literature}\label{review-of-the-literature}

\subsection{What is MTurk?}\label{what-is-mturk}

MTurk is an internet marketplace where people seeking laborers have
access to a population of workers willing to do tasks requiring a human
intelligence for a small fee (Behrend et al. (2011)). Billed as a
\enquote{marketplace for work that requires human intelligence}, MTurk
allows \enquote{requesters} to post Human Intelligence Tasks (HITs),
which serve as self-contained jobs in which \enquote{workers} can
participate. Via MTurk, researchers seeking participants often post
survey studies for workers to voluntarily complete (Goodman, Cryder, and
Cheema (2013)). Workers are given a short description about a project,
including details about various tasks, estimated time investment, and
payment, and may selectively choose which projects to complete. Here,
researchers seeking participants often post survey studies for workers
to voluntarily complete (Goodman et al. (2013)).

\subsection{Why is MTurk Valuable for Researchers Studying Working
Populations?}\label{why-is-mturk-valuable-for-researchers-studying-working-populations}

MTurk offers practical advantages that may not be available via other
more traditional methods of data collection, such as sampling from
specific organizations (Paolacci, Chandler, and Ipeirotis (2010)). The
platform allows for easier subject pool access at an affordable cost and
for subject prescreening (Mason and Suri (2012)). Additionally, it
allows for subject anonymity, subject identifiability in longitudinal
studies, and offers a simple and supportive infrastructure (Paolacci et
al. (2010)). More importantly, its diversity is a key strength that sets
it apart from other commonly studied populations (e.g., university
students, community samples), making it an attractive subject pool for
researchers to draw upon (Buhrmester, Kwang, and Gosling (2011); Erikson
\& Simpson, 2010; Paolacci et al. (2010)). For instance, Behrend et al.
(2011) found that MTurk workers are more diverse on variables such as
age, ethnicity, education, employment status, job experience,
profession, nationality, and personality. The diversity of MTurk allows
OB researchers to overcome concerns regarding the generalizability of
their findings (e.g., Paolacci et al. (2010)), which has been a key
concern facing OB research (Chiao and Cheon (2010)).

However, while researchers have praised MTurk for its diversity, few
organizational behavior researchers have considered how MTurk might be
less diverse than expected in a key respect: occupational diversity. To
our knowledge, no study has compared the occupational diversity of MTurk
to the broader US population. While Behrend et al. (2011) work suggests
that MTurk's occupational diversity is greater than the university
population, our work expands on their research by comparing MTurk's
occupational diversity to the broader US population. If the occupational
diversity of the broader US population is indeed reflected in MTurk,
findings from MTurk should better generalize across occupations.
Alternatively, if MTurk's occupational diversity over- or
under-represent certain occupations, generalizability of findings from
MTurk should be more limited.

\emph{Research Question: Is the occupational diversity of the MTurk
population representative of the broader US economy?}

\section{Methods}\label{methods}

\subsection{US Population Occupational Summary
Statistics}\label{us-population-occupational-summary-statistics}

Occupational Employment Statistics (OES) were obtained from the United
States Bureau of Labor Statistics (BLS) (2015). Occupation diversity was
measured by calculating the percentage of job holders falling under a
particular occupational category. We adopted the Intermediate
Aggregation and Standard Occupational Classification (SOC) used by the
Bureau of Labor Statistics (BLS) for this study (see table 1).
Percentages describe the percentage of workers in the US economy who
work in occupations that fall under this broader occupational category.

\section{- - - - - - - - - - - - -}\label{section}

\section{insert table 1 about here}\label{insert-table-1-about-here}

\section{- - - - - - - - - - - - -}\label{section-1}

\subsection{MTurk Occupational Summary
Statistics}\label{mturk-occupational-summary-statistics}

Participants' data were collected from two independently conducted
studies (n\textsubscript{1} = 779; n\textsubscript{2} = 1108). In one
study (n\textsubscript{1} =779), occupational data was self-reported by
participants who were asked to described their job title. They were then
brought to ONET online and asked to retrieve the six-digit occupational
classification code that most closely corresponded to their job title.
In the second study (n\textsubscript{2} = 1108), participants again
self-reported their job title. However, the six-digit occupational
classification codes were obtained by training research assistants to
match the self-reported job titles to those in the ONET database. Using
this data, we described the occupational diversity of each sample using
relative frequencies of occupational category representation as
described by Table 1.

\subsection{Analysis}\label{analysis}

A chi-square test of goodness-of-fit was used to test whether
proportions from MTurk samples deviated from expected proportions as
derived from the BLS. Specifically, we tested whether MTurk's
occupational diversity deviated (p \textless{} .05) from that expected
if we assume the BLS population parameters generalize to the MTurk
population. This tests the null hypothesis that the populations are
identical in terms of occupational diversity. To facilitate
interpretation of our data, we also caluculated 95\% confidence
intervals for our sample proportions, which allows a visual inspection
of whether or not the population value is within sampling error.

\section{Results}\label{results}

Results shown in our table indicate that MTurk's occupational frequency
distribution does not consistently align with the US economy for both
sample 1 (\(\chi^2(12) = 368.58, p < .001\)) and sample 2
(\(\chi^2(12) = 486.07, p < .001\)). Examining the 95\% confidence
intervals across both samples, certain occupations were consistently
over- or under-represented. More specifically individuals sampled from
MTurk occupied generally more white-collar jobs (e.g., Management,
Business, and Financial Occupations; Computer, Engineering, and Science
Occupations) and generally fewer blue-collar jobs (e.g., Construction
and Extraction Occupations; Installation, Maintenance, and Repair
Occupations; Production Occupations) than what would be expected based
on BLS data. For instance, 10.13\% of workers in the US economy in 2015
worked in \enquote{Management, Business, and Financial Occupations.} By
contrast, approximately 17\% and 23\% of workers from the MTurk samples
fell under this classification. These trends were generally consistent
across both study 1 and 2.

\section{- - - - - - - - - - - - -}\label{section-2}

\section{insert table 2 about here}\label{insert-table-2-about-here}

\section{- - - - - - - - - - - - -}\label{section-3}

\section{Discussion}\label{discussion}

What started our investigation was a simple research question: is the
occupational diversity of the MTurk population representative of the
broader US population? Our data suggest that the answer is not
consistently so: certain occupations are over-represented, some fall
within expectations, and still others are under-represented. Thus, OB
researchers using convenient sampling procedures that draw on the MTurk
population should consider whether occupational features might play a
role in their investigation, which is point that we will elaborate upon
in greater detail. Simply put, our results suggest that MTurk's (US)
occupational diversity does not align with the broader US economy.

\section{Recommendations for Researchers Using
MTurk}\label{recommendations-for-researchers-using-mturk}

Given that OB researchers will likely wish to continue using MTurk as a
means of gathering diverse, convenient, and low-cost survey sample data,
researchers should consider how differences in the occupational
diversity observed amongst MTurk workers relative to the US population
might impinge upon the generalizability of their findings (Newman,
Joseph, and Feitosa (2015)). Specifically, given the over-representation
of workers from (1) Management, Business, and Financial, (2) Computer,
Engineering, and Science, and (3) Education, Legal, Community Service,
Arts, and Media occupations, we might safely assume that many of our
findings from MTurk will generalize across jobs classified under these
broader over-represented occupational categories. However, as many
occupations were under-represented {[}e.g., (1) Service, (2)
Construction and Extraction, (3) Installation, Maintenance, and Repair,
(4) Production; and (5) Transportation and Material Moving{]}, questions
may arise as to whether findings will generalize to jobs falling under
these broad classifications. To address this issue, researchers should
consider capturing occupational data and modeling theoretically derived
moderators. Person-environment (P-E) fit theories should help guide
researchers in proposing how certain occupational groupings might
influence individual-level relationships, lending to more nuanced and
contextualized results (How, Meade, Behrend, and Lance (2009)). Further,
O*NET might be useful in this regard as it provides researchers a means
for quantifying key occupational differences (e.g., level of competition
requirements, social skills requirements) and thereby testing P-E fit
hypotheses pertaining to occupational group differences (see Judge and
Zapata (2015)).

\subsection{Limitations and Future
Research}\label{limitations-and-future-research}

One key limitation is that we assumed that O*NET's classification system
perfectly captured all occupations during 2015. Additionally, as both
studies involved convenient sampling procedures, it remains uncertain
the extent to which these samples adequately reflect the larger MTurk
population. Still, the consistency of our findings should be considered
by future OB/HR researchers hoping to draw on MTurk for their research
with the hopes of generalizing to the larger (or a specific) working US
population. Additionally, as our studies were conducted in the US, it
remains unclear how other MTurk populations might reflect (however
imperfectly) their respective broader economy. Future larger scale
studies should also consider lower levels of BLS occupational
classification to examine the extent to which narrower occupations might
not be represented by MTurk. Lastly, examining diversity in other
respects (age, racio-ethnic, income, education, etc.) should also be
considered. Though MTurk offers a practically feasible method of
attaining high quality data, future research will be needed to
understand the limitations of this tool.

Given that our data were gathered and analyzed in 2015, it might be
insightful to have our procedures replicated annually to examine how the
demographics of MTurk, specifically occupational characteristics, change
over time, especially as access to technology becomes even easier and
more commonplace. Thus, we encourage researchers to continue examining
the occupational diversity of MTurk relative to the US economy. We also
encourage researchers to consider novel ways of approximating the
occupational diversity of both the US economy and MTurk because the
currently existing taxonomies that we have used (i.e., BLS-based)
probably do not capture all occupations. Lastly, our results raise a new
and interesting research question: Why might MTurk possess the
occupational diversity makeup that it does (i.e., why are some
occupations over-represented while others are under-represented)?
Similarly, what factors (e.g., access to technology, interest in
psychology) might make it easier for workers coming from certain
occupations to participate in OB studies hosted on MTurk? Answering
these questions might unearth more constraints that impinge on the
findings of OB studies using MTurk.

\section{Conclusion}\label{conclusion}

We called into question the diversity of MTurk as a key strength and
suggested one key area where diversity may be lacking: occupational
diversity. Our results suggest that, compared to the broader US
population, MTurk is not as diverse, both over- and underrepresenting
certain occupations. Future research drawing on MTurk should consider
how occupational differences might be theoretically relevant and
incorporate these effects in analyses using MTurk sample data to help
overcoming these shortcomings.

\newpage

\section{References}\label{references}

\setlength{\parindent}{-0.5in} \setlength{\leftskip}{0.5in}

\hypertarget{refs}{}
\hypertarget{ref-aguinis2012conducting}{}
Aguinis, H., \& Lawal, S. O. (2012). Conducting field experiments using
eLancing's natural environment. \emph{Journal of Business Venturing},
\emph{27}(4), 493--505.

\hypertarget{ref-behrend2011viability}{}
Behrend, T. S., Sharek, D. J., Meade, A. W., \& Wiebe, E. N. (2011). The
viability of crowdsourcing for survey research. \emph{Behavior Research
Methods}, \emph{43}(3), 800.

\hypertarget{ref-bergman2016have}{}
Bergman, M. E., \& Jean, V. A. (2016). Where have all the ``workers''
gone? A critical analysis of the unrepresentativeness of our samples
relative to the labor market in the industrial--organizational
psychology literature. \emph{Industrial and Organizational Psychology},
\emph{9}(01), 84--113.

\hypertarget{ref-buhrmester2011amazon}{}
Buhrmester, M., Kwang, T., \& Gosling, S. D. (2011). Amazon's mechanical
turk a new source of inexpensive, yet high-quality, data?
\emph{Perspectives on Psychological Science}, \emph{6}(1), 3--5.

\hypertarget{ref-chiao2010weirdest}{}
Chiao, J. Y., \& Cheon, B. K. (2010). The weirdest brains in the world.
\emph{Behavioral and Brain Sciences}, \emph{33}(2-3), 88--90.

\hypertarget{ref-crump2013evaluating}{}
Crump, M. J., McDonnell, J. V., \& Gureckis, T. M. (2013). Evaluating
amazon's mechanical turk as a tool for experimental behavioral research.
\emph{PloS One}, \emph{8}(3), e57410.

\hypertarget{ref-goodman2013data}{}
Goodman, J. K., Cryder, C. E., \& Cheema, A. (2013). Data collection in
a flat world: The strengths and weaknesses of mechanical turk samples.
\emph{Journal of Behavioral Decision Making}, \emph{26}(3), 213--224.

\hypertarget{ref-griggs2016these}{}
Griggs, T. L., Eby, L. T., Maupin, C. K., Conley, K. M., Williamson, R.
L., Griek, O. H. V., \& Clauson, M. G. (2016). Who are these workers,
anyway? \emph{Industrial and Organizational Psychology}, \emph{9}(01),
114--121.

\hypertarget{ref-how2009dr}{}
How, I., Meade, A. W., Behrend, T. S., \& Lance, C. E. (2009). Dr.
strangelove, or. \emph{Statistical and Methodological Myths and Urban
Legends: Doctrine, Verity and Fable in the Organizational and Social
Sciences}, 89.

\hypertarget{ref-judge2015person}{}
Judge, T. A., \& Zapata, C. P. (2015). The person--situation debate
revisited: Effect of situation strength and trait activation on the
validity of the big five personality traits in predicting job
performance. \emph{Academy of Management Journal}, \emph{58}(4),
1149--1179.

\hypertarget{ref-landers2015inconvenient}{}
Landers, R. N., \& Behrend, T. S. (2015). An inconvenient truth:
Arbitrary distinctions between organizational, mechanical turk, and
other convenience samples. \emph{Industrial and Organizational
Psychology}, \emph{8}(02), 142--164.

\hypertarget{ref-mason2012conducting}{}
Mason, W., \& Suri, S. (2012). Conducting behavioral research on
amazon's mechanical turk. \emph{Behavior Research Methods},
\emph{44}(1), 1--23.

\hypertarget{ref-newman2015external}{}
Newman, D. A., Joseph, D. L., \& Feitosa, J. (2015). External validity
and multi-organization samples: Levels-of-analysis implications of
crowdsourcing and college student samples. \emph{Industrial and
Organizational Psychology}, \emph{8}(02), 214--220.

\hypertarget{ref-paolacci2014inside}{}
Paolacci, G., \& Chandler, J. (2014). Inside the turk: Understanding
mechanical turk as a participant pool. \emph{Current Directions in
Psychological Science}, \emph{23}(3), 184--188.

\hypertarget{ref-paolacci2010running}{}
Paolacci, G., Chandler, J., \& Ipeirotis, P. G. (2010). Running
experiments on amazon mechanical turk.

\hypertarget{ref-pathak2008methodology}{}
Pathak, R. P. (2008). \emph{Methodology of educational research}.
Atlantic Publishers \& Dist.

\hypertarget{ref-shank2016using}{}
Shank, D. B. (2016). Using crowdsourcing websites for sociological
research: The case of amazon mechanical turk. \emph{The American
Sociologist}, \emph{47}(1), 47--55.

\hypertarget{ref-simons2012common}{}
Simons, D. J., \& Chabris, C. F. (2012). Common (mis) beliefs about
memory: A replication and comparison of telephone and mechanical turk
survey methods. \emph{PloS One}, \emph{7}(12), e51876.

\hypertarget{ref-woo2015amazon}{}
Woo, S. E., Keith, M., \& Thornton, M. A. (2015). Amazon mechanical turk
for industrial and organizational psychology: Advantages, challenges,
and practical recommendations. \emph{Industrial and Organizational
Psychology}, \emph{8}(02), 171--179.

\hypertarget{ref-yarkoni2015interactions}{}
Yarkoni, T., Ashar, Y. K., \& Wager, T. D. (2015). Interactions between
donor agreeableness and recipient characteristics in predicting
charitable donation and positive social evaluation. \emph{PeerJ},
\emph{3}, e1089.


\clearpage
\renewcommand{\listtablename}{Table captions}
\listoftables




\end{document}
